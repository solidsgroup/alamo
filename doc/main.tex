\documentclass{article}
\renewcommand{\familydefault}{\sfdefault}
%\usepackage{droid}
\usepackage[cm]{sfmath}
\usepackage[labelfont=bf]{caption}
\usepackage{subcaption}
\usepackage{url}
\usepackage{bm}
\usepackage{textpos}
\usepackage{graphicx}
\usepackage[compact]{titlesec} % compact section titles
\usepackage{amsmath}
\usepackage{esint}
\usepackage{amssymb}
\usepackage{textcomp}
\usepackage{enumitem}
\usepackage{fancyhdr}
\usepackage{hyperref}\hypersetup{pdfborder={0 0 0}}
\usepackage{booktabs}
\usepackage{tabularx}
\usepackage{titlesec}
\usepackage{tocloft}
\usepackage{setspace}
\usepackage{longtable}
\usepackage{xcite}
\usepackage[table]{xcolor}
\usepackage[colorinlistoftodos,textsize=tiny]{todonotes}
\usepackage[margin=1in]{geometry}
\usepackage[backend=bibtex,natbib=true,style=numeric-comp,sorting=none]{biblatex}
\addbibresource{library.bib}
\usepackage{placeins}
\usepackage{wrapfig}
\usepackage{ifthen}
\usepackage{pdfpages}

\def\alox{Al$_2$O$_3$}

% 
% Page and paragraph formatting
% 
%\linespread{0.95}
\setlength{\parindent}{0pt}
\setlength{\parskip}{5pt}
\rhead{Brandon Scott Runnels} \chead{} \lhead{University of Colorado Colorado Springs} \rfoot{SFFP Proposal $|$ \thepage} \cfoot{} \lfoot{December 7, 2017}
\pagestyle{fancy}

% Prevent multiple float figures on the same page
\setcounter{totalnumber}{1}
\title{Working notes for Alamo development}
\author{B Runnels}
\begin{document}

\maketitle



\section{Canonical derivation of FEM for reference}

Solve
\def\C{\mathbb{C}}
\begin{gather}
  \C_{ijkl}u_{k,lj}(x) + b_i(x) = 0
\end{gather}
Discretization
\begin{align}
  u_i(x) &= u^n_i\phi^n 
  & b_i(x) = b^m_i\phi^m
\end{align}
Substitute
\begin{gather}
  \C_{ijkl}u^n_k\phi^n_{,lj}+ b_i^m\phi^m = 0
\end{gather}
Weak form
\begin{gather}
  \int_\Omega a^p\phi^p\C_{ijkl}u^n_k\phi^n_{,lj}dx + \int_\Omega a^p\phi^p \phi^m b_i^m dx = 0
\end{gather}
Integration by parts, factor out constants
\begin{gather}
  a^pu^n_k\int_\Omega \phi_{,j}^p\C_{ijkl}\phi^n_{,l}dx + a^pb_i^m\int_\Omega \phi^p \phi^m  dx = 0
\end{gather}
$\forall a^p$ to get locality
\begin{gather}
  \underbrace{\Big(\int_\Omega \phi_{,j}^p\C_{ijkl}\phi^n_{,l}dx\Big)}_{K^{pn}_{ik}}u^n_k + \underbrace{\Big(\int_\Omega \phi^p \phi^n  dx\Big)}_{M^{pm}}b_i^m = 0
\end{gather}
Stiffness matrix:
\begin{align}
  K^{pn}_{ik} = \C_{ijkl} \int_\Omega \phi_{,j}^p\phi^n_{,l}dx
\end{align}

In two dimensions -- coordinate change $x^n\to 0$. {\bf Require $\Delta x=\Delta y = \Delta$}
\begin{align}
  \phi^n 
  = \frac{1}{\Delta^2}
  \begin{cases}
    (x_1-\Delta)(x_2-\Delta) & 0<x_1<\Delta, 0<x_2<\Delta\\
    -(x_1+\Delta)(x_2-\Delta) & -\Delta <x_1<0, 0<x_2<\Delta\\
    -(x_1-\Delta)(x_2+\Delta) & 0<x_1<\Delta, -\Delta <x_2<0\\
    (x_1+\Delta)(x_2+\Delta) & -\Delta 0<x_1<0, -\Delta<x_2<0
  \end{cases}
\end{align}


For $p=n$:
\begin{align}
  \int_\Omega \phi_{,i}^p\phi^n_{,j}dx = 
  \begin{cases}
    4/3 & i=j\\
    0 & i\ne j
  \end{cases}
\end{align}

For $p$ to east or west of $n$
\begin{align}
  \int_\Omega \phi_{,i}^p\phi^n_{,j}dx = 
  \begin{cases}
    -2/3 & i=j=1\\
    1/3 & i=j=2\\
    0 & \text{else}
  \end{cases}
\end{align}

For $p$ to north or south of $n$
\begin{align}
  \int_\Omega \phi_{,i}^p\phi^n_{,j}dx = 
  \begin{cases}
    1/3 & i=j=1\\
    -2/3 & i=j=2\\
    0 & \text{else}
  \end{cases}
\end{align}

For $p$ to northeast / southwest of $n$
\begin{align}
  \int_\Omega \phi_{,i}^p\phi^n_{,j}dx = 
  \begin{cases}
    -1/6 & i=j\\
    -1/4 & i\ne j
  \end{cases}
\end{align}

For $p$ to northwest / southeast of $n$
\begin{align}
  \int_\Omega \phi_{,i}^p\phi^n_{,j}dx = 
  \begin{cases}
    -1/6 & i=j\\
    1/4 & i\ne j
  \end{cases}
\end{align}

Assuming linear isotropic plane strain:
\begin{align}
  \C_{ijkl}= \mu(\delta_{ik}\delta_{jl} +\delta_{il}\delta_{jk}) + \lambda\,\delta_{ij}\delta_{kl}
\end{align}
Substituting into stiffness matrix:
\begin{align}
  K^{pn}_{ik} = 
  \mu\,\delta_{ik}\int_\Omega \phi_{,j}^p\phi^n_{,j}dx + (\mu+\lambda)\int_\Omega \phi_{,i}^p\phi^n_{,k}dx 
  = \mu\,\delta_{ik}(\Phi^{pn}_{11}+\Phi^{pn}_{22}) + (\mu+\lambda)\Phi^{pn}_{ik}
\end{align}


\begin{tabularx}{1.0\linewidth}{XXXXX}
  \toprule
  Relative location & $i=1,k=1$ & $i=1,k=2$ & $i=2,k=1$ & $i=2,k=2$\\
  \midrule
  $n=p$ & $(2\mu+\lambda)(4/3)$ & 
\end{tabularx}


\end{document} 
